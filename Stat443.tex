\documentclass[12pt]{article}
\usepackage{mylibrary}
\usepackage{amsmath} % For math symbols
\pagestyle{fancy}
\titleformat{\chapter}[display]
  {\normalfont\huge\bfseries}{\chaptername\ \thechapter}{20pt}{\Huge\bfseries}
% Customize section title format
\titleformat{\section}
  {\normalfont\Large\bfseries}{\thesection}{1em}{}



\begin{document}
\vspace*{\fill}

\begin{center}
    \fontsize{25pt}{20pt}\selectfont
    \thispagestyle{empty}
    \textbf{Stat443}\\
    \textbf{}\\
    \text{Fall 2024}\\
    \textbf{}\\
    \textbf{}\\
\end{center}
\vspace*{\fill}
\newpage

{
    \fontsize{13pt}{18pt}\selectfont
    \tableofcontents
    
}
\newpage
\newcommand{\add}[1]{{\color{VioletRed} #1}}

%设置字号
\fontsize{12pt}{17pt}\selectfont



\newpage
\headerstyle{Chapter 1}

\mysection{Introduction to Forcasting and Time Series}{1}

\weekdate{1}{September 4th to 6th}

\begin{definition}
    Time Series Data.
    \begin{itemize}
        \item A time series is a sequence
    \end{itemize}
\end{definition}

\weekdate{2}{September 9th to 13 th}
\textbf{Mathematical Models}
\begin{itemize}
    \item The general framework is to remove the trend and seasonal patterns in the data so that we have a stationary process.
    \item The detrendized and deseasonalize data is then to be modelled.
    \item Let us formalize trend and seasonal patterns in mathematical forms.
    \item We will assume $X_t$ denotes the value of process at time t.
\end{itemize}

\begin{example}
    Models with (non-periodic) trend\\
    Consider the model $X_t = m_t + Y_t$ where $m_t$ is a slowly changing function(deterministic) and $Y_t$ is the zero-mean random component.
    \begin{itemize} 
        \item Calculate $E(X)$.
        \begin{align*}
            E(X) &= E(m_t + Y_t)\\
            &= E(m_t) + E(Y_t)\\
            &= E(m_t) \\
            &= m_t
        \end{align*}
        Therefore, the trend is ,in fact , the mean of process.
    \end{itemize}
    \item Provide some examples of $X_t$ defined above.
    $$ X_t = \alpha_0 + \alpha_1 t + \alpha_2 t^2 + Y_t$$
\end{example}

\weekdate{3}{September 16th to 20 th}


\begin{definition}
    \textbf{Properties and Statistical Inference}
\end{definition}

\begin{definition}
    \textbf{Properties of the $\Tilde{\beta}_{OLS}$($(X^TX)^{-1} X^T  y $)}
    \begin{itemize}
        \item Suppose $\epsilon \sim MVN(0,\sigma^2I_n) $, i.e. $E(\epsilon) = 0$ and $Var(\epsilon) - \sigma^2 I_n = diag(\sigma^2)$
        \item $\Tilde{\beta}_{OLS}$ is an unbiased estimator for $\beta$, i.e. $E(\Tilde{\beta}_{OLS}) = \beta$.
        \item Variance/covariance matrix of $\Tilde{\beta}_{OLS}$ is $$Var(\Tilde{\beta}_{OLS}) = Var((X^TX)^{-1}X^Ty) = ? = \sigma^2(X^TX)^{-1}$$
        \begin{itemize}
            \item $\sigma^2$ using x to predict y 
        \end{itemize}
        \item Estimate $\sigma^2$ by the following unbiased
    \end{itemize}
\end{definition}
\begin{definition}
    \textbf{ Maximum likelihood Estimation}
    \begin{itemize}
        \item The distribution of \textbf{Y} is $MVN(X\beta,\sum)$ where $\sum = \sigma^2 I_n$. The density function of \textbf{Y} is 
        \item Log-likelihood is the 
    \end{itemize}
\end{definition}


\begin{definition}
    \textbf{Sum of Squares}
    It is easy to say that

    The balancing between fit and potentially prediction power and we call this bias-variance trade-off \& alicum it later.
\end{definition}

\begin{example}
    The market value of a house is of
interest to a both buyers and sellers(n = 24). It should be a function of a
number of features of the house and a multiple linear
regression model can be used to estimate this function. In
order to calibrate the model, a data set has been constructed
using the following variables
\end{example}
\end{document}